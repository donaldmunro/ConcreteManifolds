\documentclass[11pt]{article}

\usepackage{sagetex}
\usepackage{amssymb,amsmath,mathtools,txfonts}
\usepackage[a4paper]{geometry}
\usepackage{parskip}
\setlength{\parindent}{15pt}
\usepackage[english]{babel}
\usepackage[activate={true,nocompatibility},final,tracking=true,kerning=true,spacing=true,factor=1100,stretch=10,shrink=10]{microtype}
\microtypecontext{spacing=nonfrench}
\usepackage[compact]{titlesec}
% \usepackage{enumitem}
% \setlist{nosep}
\usepackage{csquotes}
\usepackage{hyperref}
\usepackage[ backend=biber,
             style=authoryear,
             maxcitenames=2,
             mincitenames=1,
             maxnames=2,
             maxbibnames=100,
             hyperref=true,
             uniquename=false,
             backref=true,
             date=iso,
             seconds=true,
             dashed=false
           ]
            {biblatex}
\addbibresource{bibliography.bib}
\renewbibmacro{in:}{}
\setcounter{biburlnumpenalty}{100}
\setcounter{biburlucpenalty}{100}
\setcounter{biburllcpenalty}{100}
\DeclareNameAlias{author}{last-first}
\renewcommand*{\nameyeardelim}{\addcomma\space}
\DeclareFieldFormat[inbook]{citetitle}{#1}
\DeclareFieldFormat[inbook]{title}{#1}
\DeclareFieldFormat[inproceedings]{citetitle}{#1}
\DeclareFieldFormat[inproceedings]{title}{#1}
\DeclareFieldFormat[incollection]{citetitle}{#1}
\DeclareFieldFormat[incollection]{title}{#1}
\DeclareFieldFormat[article]{citetitle}{#1}
\DeclareFieldFormat[article]{title}{#1}
% \DeclareFieldFormat[article, inbook, incollection, inproceedings, misc, thesis, unpublished]{title}{#1}
\DeclareFieldFormat[article]
{volume}{ {vol. #1} }
%-- puts number/issue between brackets
\DeclareFieldFormat[article, inbook, incollection, inproceedings, misc, thesis, unpublished]
{number}{no. #1}
% {number}{\mkbibparens{#1}}
%-- and then for articles directly the pages w/o any "pages" or "pp."
\DeclareFieldFormat[article]
{pages}{ pp. #1}
%-- for some types replace "pages" by "p."
\DeclareFieldFormat[inproceedings, incollection, inbook]
{pages}{ pp. #1}
%-- format 16(4):224--225 for articles
\renewbibmacro*{volume+number+eid}{
  \printunit{\addcomma} %%% Added
  \printfield{volume}
  \printfield{number}
  % \printunit{\addcolon}
}
\DefineBibliographyStrings{english}{%
    backrefpage  = {\lowercase{c}ited on: p.}, % for single page number
    backrefpages = {\lowercase{c}ited on pp.} % for multiple page numbers
}
%-- citations with square brackets (== \usepackage[square]{natbib})
\makeatletter
\newrobustcmd*{\parentexttrack}[1]{
  \begingroup
  \blx@blxinit%
  \blx@setsfcodes%
  \blx@bibopenparen#1\blx@bibcloseparen%
  \endgroup}
\AtEveryCite{
  \let\parentext=\parentexttrack%
  \let\bibopenparen=\bibopenbracket%
  \let\bibcloseparen=\bibclosebracket}
\makeatother

\usepackage[most]{tcolorbox}
\tcbset{
    frame code={}
    center title,
    left=0pt,
    right=0pt,
    top=0pt,
    bottom=0pt,
    colback=gray!70,
    colframe=white,
    width=\dimexpr\textwidth\relax,
    enlarge left by=0mm,
    boxsep=5pt,
    arc=0pt,outer arc=0pt,
    }
\usepackage{enumitem}   
\usepackage{dirtytalk}
\usepackage{wasysym}
% \usepackage{breqn}


\begin{document}
    
    % \maketitle
    
    \hypertarget{smooth-manifolds-made-concrete-for-dumbasses}{%
\section{Smooth Manifolds made Concrete for Dumbasses}\label{smooth-manifolds-made-concrete-for-dumbasses}}

The intention behind this document is to provide an extended and comprehensive example of a smooth
manifold for dumbasses like me who can't always grok the heavily abstract without a more concrete
example, or as the OO framework types would say, a reification. The extended example is the
prototypical/canonical stereographic projection considering the sphere \(S^2\) as a manifold.
Examples cover deriving the charts, finding the inverses functions between manifolds including
deriving an ellipsoid stereographic manifold, and tangent spaces including both equivalence classes
of curves and derivation based approaches, illustrating proofs of vector space closure and change
of basis of tangent vectors. The examples are based on using the open source
\href{https://www.sagemath.org/}{SageMath} Computer Algebra System (CAS) to create an interactive
notebook. Sage is available on most Linux distributions and the web site also has a Windows
installer. The notebook has also been converted to a Latex/SageTex document which hides most of
the Sage specific manipulation and shows just the mathematics for those who aren't interested in
Sage (although you lose the ability to interact with the content including the graphs).
Note Sage does have 
\href{http://doc.sagemath.org/html/en/reference/manifolds/sage/manifolds/differentiable/manifold.html}{built in Manifold}
support, but this document won't be using it as the intention is to provide examples from first principles.
Errors or misunderstandings on my part or also more than likely as I am a computer scientist who dabbles in mathematics,
for Computer Vision, Machine Learning and Neural Networks, therefore any error corrections or suggestions 
for improvements would also be welcome.

\begin{sagesilent}
clear_vars()

def stereographic(pole):
    o_n = vector(SR, pole)
    var('x', 'y', 'z', 'u', 'v', 'l')
    assume(x, 'real', y, 'real', z, 'real', u, 'real', v, 'real', l, 'real')
#     p = vector(SR, (x, y, sqrt(x^2 + y^2)))
    p = vector(SR, (x, y, z))
    s = vector(SR, (u, v, 0))
    lhs = o_n - p
    rhs = l*(o_n - s)
    eq1 = lhs[0] == rhs[0]
    eq2 = lhs[1] == rhs[1]
    eq3 = lhs[2] == rhs[2]
    l_s = solve(eq3, l)[0].rhs()
    return (solve(eq1.substitute(l=l_s), u)[0], solve(eq2.substitute(l=l_s), v)[0])

def the_solution(sols, submap):
    for sol in sols:
        if sol.rhs().substitute(submap) != 0:
            return sol
    return None

o_n = vector(SR, (0,0,1))
var('x', 'y', 'z', 'u', 'v', 'l')
assume(x, 'real', y, 'real', z, 'real', u, 'real', v, 'real', l, 'real')
p = vector(SR, (x, y, z))
s = vector(SR, (u, v, 0))
lhs = o_n - p
rhs = l*(o_n - s)
eq1 = lhs[0] == rhs[0]
eq2 = lhs[1] == rhs[1]
eq3 = lhs[2] == rhs[2]   
\end{sagesilent}

\hypertarget{the-stereographic-projection-atlas}{%
\subsection{The Stereographic Projection
Atlas}\label{the-stereographic-projection-atlas}}

The sterographic projection is the canonical example for a
topological/smooth manifold. The example shown here will map \(S^2\) to
\(R^2\), although it can be extended to \(S^n \rightarrow R^n\).

The sphere \(S^2\) is seen as a manifold having an atlas with two open
sets and associated charts. The first chart is defined on the open set
defined by the full sphere excluding the north (top) pole \((0,0,1)\),
while the second chart is defined on a set that excludes only the south
(bottom) \((0,0,1)\) pole. Both charts project a point on the sphere
surface to the plane at \(z=0\) by extending a line from the excluded
pole through the point on the manifold (sphere) being charted onto the
plane.

To derive the chart equation let \(\vec{o_n} = (0, 0, 1)\),
\(\vec{p_n} = (x, y, z)\) be a point on the sphere and
\(\vec{s_n} = (u, v, 0)\) be the projection then (see the figure below):

\(\vec{o_n} - \vec{p_n} = l(\vec{o_n} - \vec{s_n}\))
which evaluates to: $\sage{lhs} = \sage{rhs}$
    
    Solving for the third component of the vector equation and substituting
into the first two:

\begin{sagesilent}
sol1 = solve(eq3, l)[0]
sol2 = solve(eq1, u)[0], solve(eq2, v)[0]
\end{sagesilent}
\begin{align*} 
    \sage{sol1} \\
    \sage{sol2}
\end{align*}    
    
\begin{sagesilent}
On = (0, 0, 1)
Os = (0, 0, -1)
Cn = stereographic(On)
Cs = stereographic(Os)
cn_0 = Cn[0].rhs()
cn_1 = Cn[1].rhs()
cs_0 = Cs[0].rhs()
cs_1 = Cs[1].rhs()
\end{sagesilent}

    \(\therefore\) the charts with chart coordinates given by \(u\) and \(v\) are given by.

North pole:\\

$(\sage{cn_0}, \sage{cn_1})$    
    
    South pole:\\

    $(\sage{cn_0}, \sage{cn_1})$    
    
    \hypertarget{visualization}{%
\subsection{Visualization}\label{visualization}}

To visualize the manifold and charts see Figure \ref{fig:stereo1} which demonstrates manifold points
\((\frac{1}{2}, \frac{1}{2},\frac{1}{\sqrt{2}})\) and \((\frac{1}{2}, \frac{1}{2},-\frac{1}{\sqrt{2}})\) 
projected to the plane z = 0 (it does look a lot better in the original Jupyter notebook):

\begin{sagesilent}
from sage.plot.plot3d.shapes2 import Line
def to_float(v): # Temp fix to convert to float from Sage type (see https://trac.sagemath.org/ticket/28949) for Sage 9
    vv = [None] * len(v)
    for i in range(0, len(v)):
        vv[i] = float(v[i])
    return vv    
twosq = 1/1.4142
Pn = (0.5, 0.5, twosq)
Ps = (0.5, 0.5, -twosq)
Sn = sphere(center=(0, 0, 0),size=1, color='green', aspect_ratio=[1,1,1], opacity=5/10)
Fn = plot3d(lambda x, y: 0, (-2,2), (-2,2))
on = point(On, rgbcolor=(1,0,0), size=20)
os = point(Os, rgbcolor=(0,0,0), size=20)
pn = point(Pn, rgbcolor=(1,0,0), size=20)
tpn = text3d('  (%.2f,%.2f,%.2f)'%(Pn[0], Pn[1], Pn[2]), (Pn[0], Pn[1], Pn[2]+0.2), horizontal_alignment='left',color='red', fonsize='x-small')
ps = point(Ps, rgbcolor=(0,0,0), size=20)
tps = text3d('  (%.2f,%.2f,%.2f)'%(Ps[0], Ps[1], Ps[2]), (Ps[0], Ps[1], Ps[2]-0.2), horizontal_alignment='left',color='black', fonsize='x-small')
Bn = (Cn[0].substitute({x:Pn[0], z:Pn[2]}).rhs(), Cn[1].substitute({y:Pn[1], z:Pn[2]}).rhs(), 0)
bn = point(Bn, rgbcolor=(0,0,0), size=20)
tbn = text3d('  (%.2f,%.2f)'%(Bn[0], Bn[1]), (Bn[0], Bn[1]+0.6, 0.2), horizontal_alignment='left',color='red', fonsize='x-small')
Bs = (Cs[0].substitute({x:Ps[0], z:Ps[2]}).rhs(), Cs[1].substitute({y:Ps[1], z:Ps[2]}).rhs(), 0)
bs = point(Bs, rgbcolor=(0,0,0), size=20)
#Ln = Line([On, Pn, Bn], color='red')
#Ls = Line([Os, Ps, Bs], color='black')
Ln = Line([to_float(On), to_float(Pn), to_float(Bn)], color='red')
Ls = Line([to_float(Os), to_float(Ps), to_float(Bs)], color='black')
#show(Sn+Fn+on+pn+tpn+bn+tbn+os+ps+tps+bs+Ln+Ls, figsize=8)
save(Sn+Fn+on+pn+tpn+bn+tbn+os+ps+tps+bs+Line([On, Pn, Bn], color='red')+Line([Os, Ps, Bs], color='black'), "stereo1.png")
\end{sagesilent}

\begin{figure}[ht]
    \centering
    \includegraphics[width=0.95\textwidth,keepaspectratio]{stereo1.png}
    \caption{Visualization of stereographic projection.}
    \label{fig:stereo1}
 \end{figure}
 
    \hypertarget{chart-inverses}{%
\subsection{Chart Inverses}\label{chart-inverses}}
\begin{sagesilent}
var('x', 'y', 'z', 'u', 'v', 'l')
assume(x, 'real', y, 'real', z, 'real', u, 'real', v, 'real', l, 'real')
o_n = vector(SR, On)
uv = vector(SR, (u, v, 0))
d = (o_n - uv).normalized()
Ln = o_n + d*l
\end{sagesilent}

The above charts are bijective and continuous within the domain
\(S^2 - (0, 0, 1), S^2 - (0, 0, -1)\) and codomain \((u, v) \in R^2\),
however to be valid charts they must also be bicontinuous, ie the
inverse must be continuous too, that is they must be homeomorphic to
\(R^2\).

To find the inverse first find the unit vector in the direction of the
line joining the pole to the projected coordinates, for example
\(\vec{d} = \frac{((0, 0, 1) - (u, v, 0))}{|(0, 0, 1) - (u, v, 0))|}\) = $\sage{d}$.
        
    Now express the line from the pole to the projection as a vector
equation:\\ 
\((0, 0, 1) + l \vec{d}\) = $\sage{Ln}$

\begin{sagesilent}
c_o = vector(SR, (0, 0, 0))
eq4 = (Ln - c_o).dot_product(Ln - c_o) == 1
sols = solve(eq4, l)
solution = the_solution(sols, {u:1, v:1})
In = (Ln - c_o).substitute(l=solution.rhs())
pretty_print(In)
cn_ix = In[0]
cn_iy = In[1]
cn_iz = In[2]
\end{sagesilent}
    
    The intersection of this line with the sphere occurs where
\(|(0, 0, 1) + l\vec{d} - (0, 0, 0)| = 1\), or more generally
\(|(0, 0, 1) + l\vec{d} - \vec{c}| = r\) where \(\vec{c}\) is the sphere
centre and r is the radius. This can also be expressed as
\([(0, 0, 1) + l\vec{d} - (0, 0, 0)] \cdot [(0, 0, 1) + l\vec{d} - (0, 0, 0)] = r\):\\
$\sage{eq4}$
    
    This quadratic will have one solution for \(l = 0\) at the pole. The
other solution will be the required \((x, y, z)\) on the sphere
manifold. Letting Sage do the PT:\\
% \begin{tcolorbox}
% sols = solve(eq4, l)\\
% \end{tcolorbox}
$\sage{sols}$

    and substituting the appropriate \(l\) back into the line equation:\\
 $\sage{In}$   
    
    See also the
\href{https://en.wikipedia.org/wiki/Line\%E2\%80\%93sphere_intersection}{Wikipeadia}
page on sphere-line intersection (note their notation swops \(\vec{d}\)
and \(l\) as used in the above).

\begin{sagesilent}
o_s = vector(SR, Os)
d = (o_s - uv).normalized()
Ls = o_s + d*l
eq5 = (Ls - c_o).dot_product(Ls - c_o) == 1
sols = solve(eq5, l)
solution = the_solution(sols, {u:1, v:1})        
Is = (Ls - c_o).substitute(l=solution.rhs())
cs_ix = Is[0]
cs_iy = Is[1]
cs_iz = Is[2]    
\end{sagesilent}
    
    Similarly the inverse for the south pole chart is given by (it could
probably also be derived by symmetry):\\
$\sage{Is}$
    
    \hypertarget{verifying-the-charts}{%
\subsection{Verifying the Charts}\label{verifying-the-charts}}
\begin{sagesilent}
    uu = cs_0.substitute({x: cn_ix, z: cn_iz}).full_simplify()
    vv = cs_1.substitute({y: cn_iy, z: cn_iz}).full_simplify()
    pretty_print("(", uu, ", ", vv, ")")    
\end{sagesilent}

    The two charts described above are homeomorphic and are also
differentiable with a differentiable inverse, therefore they are
diffeomorphic too. Assume the charts are named \(c_n\) and \(c_s\), then
to verify the two charts define a compatible atlas it is required that
the intersection of their domain open sets be empty ie
\(S^2 - (0, 0, 1) \cap S^2 - (0, 0, -1) = \emptyset,\) (definitely not
the case) or
\(c_s \circ c_n^{-1} : c_n(S^2 - (0, 0, 1) \cap S^2 - (0, 0, -1)) \rightarrow c_s(S^2 - (0, 0, 1) \cap S^2 - (0, 0, -1))\)
and the opposite (\(c_n \circ c_s^{-1}\)) is diffeomorphic. After
performing the composition we get:\\
$(\sage{uu}, \sage{vv})$

    which is homeomorphic and smooth over the specified domain.

    \hypertarget{smooth-functions-between-manifolds}{%
\subsection{Smooth Functions between
Manifolds}\label{smooth-functions-between-manifolds}}

A smooth map \(F: M \rightarrow N\) can be described \parencite{lee} as
\say{if for every \(p \in M\), there exist smooth charts
\((U, \phi)\) containing \(p\) and \((V, \psi)\) containing \(F(p)\)
such that \(F(U) \subseteq V\) and the composite map
\(\psi \circ F \circ \phi^{-1}\) is smooth from \(\phi(U)\) to
\(\psi(V)\)}.

We will start by defining a manifold E on an ellipsoid to use as the
destination manifold corresponding to N above for the map. U will then
correspond to either \(S^2 - (0, 0, 1)\) or \(S^2 - (0, 0, -1)\)
depending on the source chart selection. N will similarly correspond to
\(E - (0, 0, 1)\) or \(E - (0, 0, -1)\). Because E uses $ (0, 0, 1)$
and $(0, 0, -1)$ the a and b ellipse axes are arbitarily defined but
\(c = 1\) (the z semi-axis) to ensure that the poles are at
\((0, 0, 1)\) and \((0, 0, -1)\).

\hypertarget{the-ellipsoid-as-a-stereographical-manifold}{%
\subsubsection{The Ellipsoid as a Stereographical Manifold}\label{the-ellipsoid-as-a-stereographical-manifold}}
\begin{sagesilent}
en_0 = cn_0
en_1 = cn_1
es_0 = cs_0
es_1 = cs_1
var('x', 'y', 'z', 'u', 'v', 'l', 'a', 'b', 'c')
assume(x, 'real', y, 'real', z, 'real', u, 'real', v, 'real', l, 'real', a, 'real', b, 'real', c, 'real')
o_n = vector(SR, (0, 0, 1))
uv = vector(SR, (u, v, 0))
d = (o_n - uv).normalized()
Ln = o_n + d*l - vector(SR, (0, 0, 0))
\end{sagesilent}
    
The chart map from the ellipsoid to \(R^2\) can be derived similarly to
the sphere,resulting in \(u=-\frac{x}{z-1}, v=-\frac{y}{z-1}\) for the
north chart and \(u=\frac{x}{z+1}, v=\frac{y}{z+1}\), but with \(z\)
calculated for the ellipse. The inverse is a little more problematic
however. As with the sphere start by describing the line from the pole
to the projection in terms of a unit vector pointing in the direction of
the line:
\begin{equation}
\label{eq:ellipsoid-line}
\scriptstyle
L = (0, 0, 1) + l \frac{(0, 0, 1) - (u, v, 0)}{|(0, 0, 1) - (u, v, 0)|} = \sage{Ln}
\end{equation}
    
    Recall that a vector equation for an ellipsoid can be
\href{https://en.wikipedia.org/wiki/Ellipsoid\#In_general_position}{defined}
as \((\mathbf{x}-\mathbf{o})^{\mathrm{T}} A(\mathbf{x}-\mathbf{o})=1\)
where \(0\) is the origin coordinates and A is a positive definite
matrix which in the general case may include rotation, but in our case
contains only the ellipsoid axes:\\
\begin{sagesilent}
A = matrix(SR, 3, 3, [1/a^2, 0, 0,  0, 1/b^2, 0,   0, 0, 1]) # c = 1 to make (0,0,1) possible
el = Ln.row()*A*Ln
eq1 = el[0] == 1
sols = solve(eq1, l)
\end{sagesilent}
$\sage{A}$
    
The ellipsoid equation is then \(L^T A L\):\\
$\sage{eq1}$
    
with two solutions again with one at 0 corresponding to the pole:\\
$\sage{sols}$
    
Substituting back into \(L\) (Equation \ref{eq:ellipsoid-line}):\\
\begin{sagesilent}
solution = the_solution(sols, {u:1, v:1, a:1, b:1})        
EIn = Ln.substitute(l=solution.rhs())
en_ix = EIn[0]
en_iy = EIn[1]
en_iz = EIn[2]

o_s = vector(SR, (0, 0, -1))
d = (o_s - uv).normalized()
Ls = o_s + d*l - vector(SR, (0, 0, 0))
el = Ls.row()*A*Ls
eq1 = el[0] == 1
sols = solve(eq1, l)
solution = the_solution(sols, {u:1, v:1, a:1, b:1})          
EIs = Ls.substitute(l=solution.rhs())
es_ix = EIs[0]
es_iy = EIs[1]
es_iz = EIs[2]
\end{sagesilent}
$\sage{EIn}$
    
Similarly for the south pole we get:\\
$\sage{EIs}$
    
    \hypertarget{defining-the-inter-manifold-function}{%
\subsection{Defining the Inter-Manifold
Function}\label{defining-the-inter-manifold-function}}
\begin{sagesilent}
T = matrix(SR, 3, 3, [a, 0, 0,  0, b, 0,   0, 0, 1]) 
ev = vector(SR, (x, y, sqrt(1 - x^2 - y^2)))
F = T*ev
Fx = F[0]
Fy = F[1]
Fz = F[2]    
Ellips = F.substitute({x: cn_ix, y: cn_iy})
# pretty_print(Ellips)
f1 = en_0.substitute({x:Ellips[0], z:Ellips[2]}).full_simplify()
f2 = en_1.substitute({y:Ellips[1], z:Ellips[2]}).full_simplify()
\end{sagesilent}

A linear transformation applied to a sphere results in an ellipse. In
this case we use a linear transformation matrix:\\
$\sage{T}$

Applied to $\sage{ev}$  resulting in \(F: S \rightarrow E\) where S is the sphere manifold and E
is the ellipsoid one:\\
$F = \sage{F}$
    
    Let the ellipse chart be \((V, e_n): V \subset E\) (or
\((V, e_s): V \subset E\)) and the sphere chart be
\((U, s_n): U \subset S\) (or \((U, s_s): U \subset S\)), then

\(e_n \circ F \circ s_n^{-1}[s_n(U)]\)

should be a smooth function from \(R^2\) (the stereographic projection
coordinates for the sphere) to \(R^2\) (the stereographic projection
coordinates for the ellipse) (its probably best to take Sage's word for
it unless you \textbf{really} enjoy Algebra):\\
$\Bigg(\sage{f1}, \sage{f2}\Bigg)$
    
    \hypertarget{tangent-vectors-and-spaces}{%
\subsection{Tangent Vectors and
Spaces}\label{tangent-vectors-and-spaces}}

Confusingly there are three different approaches to defining tangent
spaces on manifolds \parencite{montgomery}. These examples will
cover the geometrically based curve approach and the more algebraic
derivation approach. 

\subsubsection{Tangent Spaces defined as Equivalence class of Curves}
 This approach starts with a curve with a domain on an
interval of \(R^1\) and the manifold \(M\) as the co-domain i.e
\(\lambda: I \subset R^1 \rightarrow M\). The simplest approach combines the
co-domain of the curve with a chart map to define curves as being
tangent at a point \(p\) if \parencite[see][pp. 73]{isham}: 
\begin{enumerate}
    \item Values on the manifold correspond ie \(\lambda_{1}(t_0) = \lambda_{2}(t_1) = p\),
    \item The velocities at the point in a chart are the same ie
    \((c_n \circ \lambda_{1})^\prime\lvert_{t=t_0} = (c_n \circ \lambda_{2})^\prime\lvert_{t=t_1}\)
\end{enumerate}

The curve used in the example is a spiral on \(S^2\) (from
\href{https://math.stackexchange.com/questions/140541/finding-parametric-curves-on-a-sphere}{math.stackexchange})

\(\lambda_1 = (x, y, z): (\sqrt{1-t^{2}} \cos (a \pi t), \sqrt{1-t^{2}} \sin (a \pi t), t)\)

To find the geometric tangent vector at a point we use the north pole
chart to project a point from the manifold specified by the curve to the
\(R^2\) Euclidean space of the stereographic projection:\\
\begin{sagesilent}
var('s', 't', 'p_4', 'a', 'pt')
a = 3
pt = 4/5
p_4 = pi/a
K_1 = (sqrt(1-t*t)*cos(7*pi*t), sqrt(1-t*t)*sin(7*pi*t), t)
T1 = vector(SR, (cn_0.substitute({x:K_1[0], z:K_1[2]}), cn_1.substitute({y:K_1[1], z:K_1[2]})))
V = T1.diff(t)
\end{sagesilent}
As a reminder, the north chart: $(\sage{cn_0}, \sage{cn_1})$\\
$(c_n \circ \lambda_1) = \sage{T1}$
    
Next we differentiate the above to get the tangent vector:\\
$\scriptstyle \sage{V}$
    
Next to find another tangent vector with the necessary properties
(having the same value in the manifold at \(t\) and equal velocity
(derivitive) in the chart), we find the inverse of the tangent line in
the chart which results in a circle in the manifold (\(s\) is the
parametric variable for the tangent line/circle) 
$\sage{T1.substitute(t=pt).n(digits=4)} + s \sage{V.substitute(t=pt).n(digits=4)}$:\\
\begin{sagesilent}
Tn = T1.substitute(t=pt) + V.substitute(t=pt)*s # Tangent line in projection
Tm = vector(SR, (cn_ix.substitute(u=Tn[0], v=Tn[1]), cn_iy.substitute(u=Tn[0], v=Tn[1]), cn_iz.substitute(u=Tn[0], v=Tn[1]))).simplify_full()
\end{sagesilent}
$\lambda_2 = \sage{Tm}$
    
Plotting the results on the manifold (Figure \ref{fig:stereo2}):\\
\begin{sagesilent}
Sn = sphere(center=(0, 0, 0),size=1, color='green', aspect_ratio=[1,1,1], opacity=0.7)
curv_1 = parametric_plot3d(K_1, (t,-1,1), color='red')
curv_2 = parametric_plot3d(Tm, (s,-1,1), color='blue')
Pn = (K_1[0].substitute(t=pt).n(), K_1[1].substitute(t=pt).n(), K_1[2].substitute(t=pt).n())
#Pn = (K_2[0].substitute(t=iss).n(), K_2[1].substitute(t=iss).n(), K_2[2].substitute(t=iss).n())
pn = point(Pn, rgbcolor=(1,0,0), size=20)
save(Sn+curv_1+curv_2+pn, "stereo2.png")
\end{sagesilent}

\begin{figure}[ht]
    \centering
    \includegraphics[width=0.95\textwidth,keepaspectratio]{stereo2.png}
    \caption{Visualization of tangent vector curves}
    \label{fig:stereo2}
 \end{figure}
    
    Verifying the two curves satisfy the equivalence class criteria:\\
    \textbf{Value on Manifold}:\\
    $(\sage{K_1[0].substitute(t=pt).n(digits=7)}, \sage{K_1[1].substitute(t=pt).n(digits=7)}, \sage{K_1[2].substitute(t=pt).n(digits=7)})$\\
    $(\sage{Tm.substitute(s=0).n(digits=7)})$\\
    \textbf{Velocity in chart}:\\
    $\sage{V.substitute(t=pt).n(digits=7)}$\\
    $\sage{Tn.diff(s).substitute(s=0).n(digits=7)}$\\
    
    \hypertarget{derivation-based-approach}{%
\subsection{Derivation based Approach}\label{derivation-based-approach}}

A derivation at a point in a \(C^\infty\) manifold is defined to be a
linear map \(D(C^\infty) \rightarrow R\),

\((D_{1}+D_{2})(f) = D_{1}(f)+D_{2}(f)\)

\((\lambda D)(f) = \lambda D(f)\) which also obeys the Leibnitz identity
\(D(f g)=D(f) \cdot g(x)+f(x) \cdot D(g)\).

Computer science types could see the derivation as defining an abstract
interface, with the actual derivitive being a concrete class
implementing the abstract interface.

See \textcite{parzygnat} (lecture 12 part 3) for a description of
derivations and associated associative algebra (in a Euclidean manifold
context, but most of what is covered also applies to general manifolds,
other than the manifold chart map directions are inverted).

A tangent vector can then be applied to a \(C^\infty\) function \(f\) given a curve \(\lambda\) by
\(D(f \circ \lambda)[t]\) where \(\lambda(t) = \vec{p} \in M\) where \(\vec{p}\) is the point in
the manifold where the tangent is taken. The derivation based velocities defined this way should
form a vector space similarly to the previous section tangent vectors which can fairly easily be
shown to form a vector space by adding/multiplying in chart space \parencite[see][pp. 76]{isham} .
Proving a vector space using the \(D(f \circ \lambda)[t]\) approach is a little more difficult.
Additive closure is covered quite lengthily in lesson 9 of \textcite{XylyXylyX} and more concisely
by \textcite{schuller} lecture 5, while Schuller also covers scalar multiplication.

The above proofs are regurgitated here for documentary purposes as I
couldn't find a book containing the proofs. Intermediate results in the
proofs are also illustrated using the stereographic example.

Multivariable differentation is denoted, as for derivations, with a \(D\) operator. For
multivariable functions and maps the differentation result will be a Jacobian, and in many cases
with tangent spaces mapping to \(R^1\) will be vectors, although transition maps between charts for
example will be matrices. In some cases the differention operator \(D\) will be subscripted with
the variables w.r.t. which the differentiation is taking place eg \(D_t\). This is in contrast with
many physics books which use the old fashioned multiple different chain rules to extensively
utilize component indexing, instead of one chain rule using linear transformations with Jacobian
matrices applying in all cases (as for example in \textcite{spivak} or more colorfully in
\textcites{ghrist}{ghrist-youtube}). Where applicable the comparable `standard' notation will also
be shown.

We will continue to use the curves defined in the previous section. We
also define \(f(\vec{p}) : S \rightarrow R = 2x + 2y + 2z\) (where z is
constrained to be on the manifold ie \(z = \sqrt{1 - x^2 - y^2}\)). Now
\(D(f \circ \lambda_1)\) and \(D(f \circ \lambda_2)\) are velocities:\\
\begin{sagesilent}
var('x', 'y', 'z')
f = 2*x + 2*y + 2*z
d1 = f.substitute(x = K_1[0], y = K_1[1], z = K_1[2]).diff(t)
d2 = f.substitute(x = Tm[0], y = Tm[1], z = Tm[2]).diff(s) #.full_simplify()
#print(latex(d2)) # Manually split the output
\end{sagesilent}

$\scriptstyle (f \circ \lambda_1)^\prime = \sage{d1}$
\begin{multline*}
    \scriptstyle
    (f \circ \lambda_2)^\prime = 
    \frac{4 \, {({(3969 \, \pi^{2} + 625)} s + 225)}}{{(3969 \, \pi^{2} + 625)} s^{2} + 450 \, s + 90} - 
    \frac{4 \, {({(3969 \, \pi^{2} + 625)} s^{2} + 450 \, s + 72)} {({(3969 \, \pi^{2} + 625)} s + 225)}}{{({(3969 \, \pi^{2} + 625)} s^{2} + 450 \, s + 90)}^{2}} -\\
    \scriptstyle
    \frac{6 \, {(63 \, \pi s {(\sqrt{5} - 1)} - {(25 \, s + 9)} \sqrt{2 \, \sqrt{5} + 10})} {({(3969 \, \pi^{2} + 625)} s + 225)}}{{({(3969 \, \pi^{2} + 625)} s^{2} + 450 \, s + 90)}^{2}} - 
    \frac{6 \, {(63 \, \pi s \sqrt{2 \, \sqrt{5} + 10} + 25 \, s {(\sqrt{5} - 1)} + 9 \, \sqrt{5} - 9)} {({(3969 \, \pi^{2} + 625)} s + 225)}}{{({(3969 \, \pi^{2} + 625)} s^{2} + 450 \, s + 90)}^{2}} +\\ 
    \scriptstyle
    \frac{3 \, {(63 \, \pi {(\sqrt{5} - 1)} - 25 \, \sqrt{2 \, \sqrt{5} + 10})}}{{(3969 \, \pi^{2} + 625)} s^{2} + 450 \, s + 90} +
    \frac{3 \, {(63 \, \pi \sqrt{2 \, \sqrt{5} + 10} + 25 \, \sqrt{5} - 25)}}{{(3969 \, \pi^{2} + 625)} s^{2} + 450 \, s + 90}
\end{multline*}
    
    To demonstrate the curves form a vector space (for this example anyway), before providing a proof (of closure under addition and scalar multiplication
    - the rest is left as an exercise for the reader \smiley{}) we first set  $\lambda_3 = \lambda_1 + \lambda_2$ and then substituting the values for the 
    intersection point specified by $s = 0$ and $t = \frac{4}{5}$ into the three different velocities.\\
\begin{sagesilent}
K_3 = vector(SR, ( (K_1[0] + Tm[0]).full_simplify(), (K_1[1] + Tm[1]), (K_1[2] + Tm[2]) ) ).simplify_full()
d3 = f.substitute(x = K_3[0], y = K_3[1], z = K_3[2]).diff(s).full_simplify()
\end{sagesilent}   
$(D(f \circ \lambda_1)[\sage{pt}] = \sage{d1.substitute(t=pt).n(digits=7)}$\\
$(D(f \circ \lambda_2)[0] = \sage{d2.substitute(s=0).n(digits=7)}$\\
$(D(f \circ \lambda_3)[\sage{pt}] = \sage{d3.substitute(s=0, t=pt).n(digits=7)}$\\
% \begin{multline*}
% \scriptscriptstyle
% \lambda_3 = -\frac{750141 \, \sqrt{5} \pi^{3} s^{2}}{15752961 \, \pi^{4} s^{4} + 4961250 \, \pi^{2} s^{4} + 3572100 \, \pi^{2} s^{3} + 714420 \, \pi^{2} s^{2} + 390625 \, s^{4} + 562500 \, s^{3} + 315000 \, s^{2} + 81000 \, s + 8100} -\\ 
% \scriptscriptstyle \frac{750141 \, \pi^{3} s^{2} \sqrt{2 \, \sqrt{5} + 10}}{15752961 \, \pi^{4} s^{4} + 4961250 \, \pi^{2} s^{4} + 3572100 \, \pi^{2} s^{3} + 714420 \, \pi^{2} s^{2} + 390625 \, s^{4} + 562500 \, s^{3} + 315000 \, s^{2} + 81000 \, s + 8100} +\\
% \scriptscriptstyle \frac{750141 \, \pi^{3} s^{2}}{15752961 \, \pi^{4} s^{4} + 4961250 \, \pi^{2} s^{4} + 3572100 \, \pi^{2} s^{3} + 714420 \, \pi^{2} s^{2} + 390625 \, s^{4} + 562500 \, s^{3} + 315000 \, s^{2} + 81000 \, s + 8100} -\\
% \scriptscriptstyle \frac{297675 \, \sqrt{5} \pi^{2} s^{2}}{15752961 \, \pi^{4} s^{4} + 4961250 \, \pi^{2} s^{4} + 3572100 \, \pi^{2} s^{3} + 714420 \, \pi^{2} s^{2} + 390625 \, s^{4} + 562500 \, s^{3} + 315000 \, s^{2} + 81000 \, s + 8100} +\\ 
% \scriptscriptstyle \frac{297675 \, \pi^{2} s^{2} \sqrt{2 \, \sqrt{5} + 10}}{15752961 \, \pi^{4} s^{4} + 4961250 \, \pi^{2} s^{4} + 3572100 \, \pi^{2} s^{3} + 714420 \, \pi^{2} s^{2} + 390625 \, s^{4} + 562500 \, s^{3} + 315000 \, s^{2} + 81000 \, s + 8100} +\\ 
% \scriptscriptstyle \frac{297675 \, \pi^{2} s^{2}}{15752961 \, \pi^{4} s^{4} + 4961250 \, \pi^{2} s^{4} + 3572100 \, \pi^{2} s^{3} + 714420 \, \pi^{2} s^{2} + 390625 \, s^{4} + 562500 \, s^{3} + 315000 \, s^{2} + 81000 \, s + 8100} -\\ 
% \scriptscriptstyle \frac{214326 \, \sqrt{5} \pi^{2} s}{15752961 \, \pi^{4} s^{4} + 4961250 \, \pi^{2} s^{4} + 3572100 \, \pi^{2} s^{3} + 714420 \, \pi^{2} s^{2} + 390625 \, s^{4} + 562500 \, s^{3} + 315000 \, s^{2} + 81000 \, s + 8100} -\\ 
% \scriptscriptstyle \frac{118125 \, \sqrt{5} \pi s^{2}}{15752961 \, \pi^{4} s^{4} + 4961250 \, \pi^{2} s^{4} + 3572100 \, \pi^{2} s^{3} + 714420 \, \pi^{2} s^{2} + 390625 \, s^{4} + 562500 \, s^{3} + 315000 \, s^{2} + 81000 \, s + 8100} +\\ 
% \scriptscriptstyle \frac{214326 \, \pi^{2} s \sqrt{2 \, \sqrt{5} + 10}}{15752961 \, \pi^{4} s^{4} + 4961250 \, \pi^{2} s^{4} + 3572100 \, \pi^{2} s^{3} + 714420 \, \pi^{2} s^{2} + 390625 \, s^{4} + 562500 \, s^{3} + 315000 \, s^{2} + 81000 \, s + 8100} -\\ 
% \scriptscriptstyle \frac{118125 \, \pi s^{2} \sqrt{2 \, \sqrt{5} + 10}}{15752961 \, \pi^{4} s^{4} + 4961250 \, \pi^{2} s^{4} + 3572100 \, \pi^{2} s^{3} + 714420 \, \pi^{2} s^{2} + 390625 \, s^{4} + 562500 \, s^{3} + 315000 \, s^{2} + 81000 \, s + 8100} +\\
% \scriptscriptstyle \frac{500094 \, \pi^{2} s}{15752961 \, \pi^{4} s^{4} + 4961250 \, \pi^{2} s^{4} + 3572100 \, \pi^{2} s^{3} + 714420 \, \pi^{2} s^{2} + 390625 \, s^{4} + 562500 \, s^{3} + 315000 \, s^{2} + 81000 \, s + 8100} +\\
% \scriptscriptstyle \frac{118125 \, \pi s^{2}}{15752961 \, \pi^{4} s^{4} + 4961250 \, \pi^{2} s^{4} + 3572100 \, \pi^{2} s^{3} + 714420 \, \pi^{2} s^{2} + 390625 \, s^{4} + 562500 \, s^{3} + 315000 \, s^{2} + 81000 \, s + 8100} -\\ 
% \scriptscriptstyle \frac{46875 \, \sqrt{5} s^{2}}{15752961 \, \pi^{4} s^{4} + 4961250 \, \pi^{2} s^{4} + 3572100 \, \pi^{2} s^{3} + 714420 \, \pi^{2} s^{2} + 390625 \, s^{4} + 562500 \, s^{3} + 315000 \, s^{2} + 81000 \, s + 8100} +\\ 
% \scriptscriptstyle \frac{46875 \, s^{2} \sqrt{2 \, \sqrt{5} + 10}}{15752961 \, \pi^{4} s^{4} + 4961250 \, \pi^{2} s^{4} + 3572100 \, \pi^{2} s^{3} + 714420 \, \pi^{2} s^{2} + 390625 \, s^{4} + 562500 \, s^{3} + 315000 \, s^{2} + 81000 \, s + 8100} +\\ 
% \scriptscriptstyle \frac{46875 \, s^{2}}{15752961 \, \pi^{4} s^{4} + 4961250 \, \pi^{2} s^{4} + 3572100 \, \pi^{2} s^{3} + 714420 \, \pi^{2} s^{2} + 390625 \, s^{4} + 562500 \, s^{3} + 315000 \, s^{2} + 81000 \, s + 8100} +\\ 
% \scriptscriptstyle \frac{17010 \, \sqrt{5} \pi}{15752961 \, \pi^{4} s^{4} + 4961250 \, \pi^{2} s^{4} + 3572100 \, \pi^{2} s^{3} + 714420 \, \pi^{2} s^{2} + 390625 \, s^{4} + 562500 \, s^{3} + 315000 \, s^{2} + 81000 \, s + 8100} -\\ 
% \scriptscriptstyle \frac{33750 \, \sqrt{5} s}{15752961 \, \pi^{4} s^{4} + 4961250 \, \pi^{2} s^{4} + 3572100 \, \pi^{2} s^{3} + 714420 \, \pi^{2} s^{2} + 390625 \, s^{4} + 562500 \, s^{3} + 315000 \, s^{2} + 81000 \, s + 8100} +\\ 
% \scriptscriptstyle \frac{17010 \, \pi \sqrt{2 \, \sqrt{5} + 10}}{15752961 \, \pi^{4} s^{4} + 4961250 \, \pi^{2} s^{4} + 3572100 \, \pi^{2} s^{3} + 714420 \, \pi^{2} s^{2} + 390625 \, s^{4} + 562500 \, s^{3} + 315000 \, s^{2} + 81000 \, s + 8100} +\\ 
% \scriptscriptstyle \frac{33750 \, s \sqrt{2 \, \sqrt{5} + 10}}{15752961 \, \pi^{4} s^{4} + 4961250 \, \pi^{2} s^{4} + 3572100 \, \pi^{2} s^{3} + 714420 \, \pi^{2} s^{2} + 390625 \, s^{4} + 562500 \, s^{3} + 315000 \, s^{2} + 81000 \, s + 8100} -\\ 
% \scriptscriptstyle \frac{17010 \, \pi}{15752961 \, \pi^{4} s^{4} + 4961250 \, \pi^{2} s^{4} + 3572100 \, \pi^{2} s^{3} + 714420 \, \pi^{2} s^{2} + 390625 \, s^{4} + 562500 \, s^{3} + 315000 \, s^{2} + 81000 \, s + 8100} +\\ 
% \scriptscriptstyle \frac{78750 \, s}{15752961 \, \pi^{4} s^{4} + 4961250 \, \pi^{2} s^{4} + 3572100 \, \pi^{2} s^{3} + 714420 \, \pi^{2} s^{2} + 390625 \, s^{4} + 562500 \, s^{3} + 315000 \, s^{2} + 81000 \, s + 8100} -\\ 
% \scriptscriptstyle \frac{5400 \, \sqrt{5}}{15752961 \, \pi^{4} s^{4} + 4961250 \, \pi^{2} s^{4} + 3572100 \, \pi^{2} s^{3} + 714420 \, \pi^{2} s^{2} + 390625 \, s^{4} + 562500 \, s^{3} + 315000 \, s^{2} + 81000 \, s + 8100} +\\ 
% \scriptscriptstyle \frac{5400 \, \sqrt{2 \, \sqrt{5} + 10}}{15752961 \, \pi^{4} s^{4} + 4961250 \, \pi^{2} s^{4} + 3572100 \, \pi^{2} s^{3} + 714420 \, \pi^{2} s^{2} + 390625 \, s^{4} + 562500 \, s^{3} + 315000 \, s^{2} + 81000 \, s + 8100} +\\ 
% \scriptscriptstyle \frac{21600}{15752961 \, \pi^{4} s^{4} + 4961250 \, \pi^{2} s^{4} + 3572100 \, \pi^{2} s^{3} + 714420 \, \pi^{2} s^{2} + 390625 \, s^{4} + 562500 \, s^{3} + 315000 \, s^{2} + 81000 \, s + 8100}
% \end{multline*}

    \hypertarget{additive-closure-proof}{%
\subsubsection{Additive Closure Proof}\label{additive-closure-proof}}

The proof starts with a construction, which given two existing curves
\(\lambda_1,\lambda_2\) intersecting at point \(\vec{p}\) at parameter
values \(t_1\) and \(t_2\), constructs a third curve \(\lambda_3\) also
intersecting \(\vec{p}\) at a parameter \(t_3\) ie
\(\lambda_1(t_1) = \lambda_2(t_2) = \lambda_3(t_3) = \vec{p}\):

\(\lambda_3(t) = c^{-1}\{ (c \circ \lambda_1)[t_1 + t] + (c \circ \lambda_2)[t_2 + t] - (c \circ \lambda_1)[t_1]\}\)

At \(t = 0\) this evaluates to
\(c^{-1}\{ c(\vec{p}) + c(\vec{p}) - c(\vec{p}) \} = c^{-1}(c(\vec{p})) = \vec{p}\)
so \(t_3 = 0\). Checking this construct results in the same values as
the other two curves:\\
\begin{sagesilent}
t1 = vector(SR, (cn_0.substitute(x=K_1[0], z=K_1[2]).substitute(t=t+pt), cn_1.substitute(y=K_1[1], z=K_1[2]).substitute(t=t+pt)))
t2 = vector(SR, (cn_0.substitute(x=Tm[0], z=Tm[2]), cn_1.substitute(y=Tm[1], z=Tm[2]))).substitute(s=t+0)
t3 = vector(SR, (cn_0.substitute(x=K_1[0], z=K_1[2]), cn_1.substitute(y=K_1[1], z=K_1[2]))).substitute(t=pt)
tt = t1 + t2 - t3
K3 = vector(SR, (cn_ix.substitute(u=tt[0], v=tt[1]), cn_iy.substitute(u=tt[0], v=tt[1]), cn_iz.substitute(u=tt[0], v=tt[1]))).simplify_full()
\end{sagesilent}
$\textbf{Value on Manifold:}~~" \sage{K3.substitute(t=0).n(digits=7)}$
    
    Because the third curve parameter \(t_3\) evaluates to 0, we can
therefore write the velocity vector for \(\lambda_3\) at \(\vec{p}\) as
\((f \circ \lambda_3)^\prime[t_3 = 0]\).

Inserting a chart and its inverse and using associativity results in

\(\{(f \circ c^{-1}) \circ (c \circ \lambda_3)\}^\prime[0]\).

Using the chain rule with the \(f \circ c^{-1}\) as the outer function:

\(D(f \circ c^{-1})[(c \circ \lambda_3)[0])~D(c \circ \lambda_3)[0]\)

Substituting the expression for \(\lambda_3\) into the right side term:

\(D(c \circ \lambda_3)[0] = D(c \circ c^{-1}\{ (c \circ \lambda_1)[t_1 + t] + (c \circ \lambda_2)[t_2 + t] - (c \circ \lambda_1)[t_1]\}) =\)

\(D((c \circ c^{-1} \circ c \circ \lambda_1))[t_1 + t] + (c \circ c^{-1} \circ c \circ \lambda_2)[t_2 + t]) - (c \circ c^{-1} \circ c \circ \lambda_1))[t_1])\)
(distributivity, associativity)

The third term is not dependent on \(t\) so the derivitive would be
\(0 \therefore\) is will disappear and the inner \(c^{-1} \circ c\) in
the other terms will cancel leaving:

\(D((c \circ \lambda_1))[t_1 + t] + (c \circ \lambda_2)[t_2 + t]))[0] = D(c \circ \lambda_1)[t_1] + D(c \circ \lambda_2)[t_2]\)

    The left hand term can also be simplified:

\(D(f \circ c^{-1})[(c \circ \lambda_3)[0]) \equiv D(f \circ c^{-1})[(c \circ \lambda_3[0])) = D(f \circ c^{-1})[c(\vec{p})]\)

as \(\lambda_3[0] = \vec{p}\) by construction.

Recombining the simplified left and right sides:

\begin{equation}
\label{eq:addition-close}    
\scriptstyle
D(f \circ c^{-1})[c(\vec{p})] \cdot (D(c \circ \lambda_1)[t_1] + D(c \circ \lambda_2)[t_2]) = D(f \circ c^{-1})[c(\vec{p})] D(c \circ \lambda_1)[t_1] + D(f \circ c^{-1})[c(\vec{p})] D(c \circ \lambda_2)[t_2])
\end{equation}

Using the chain rule to differentiate \(D(f \circ \lambda_1)[t_1]\) to show that it is the same as the first term above:

\(D(f \circ \lambda_1)[t_1] = D((f \circ c^{-1}) \circ (c \circ \lambda_1))[t_1] =\)

\(D(f \circ c^{-1})[(c \circ \lambda_1)[t_1]] \cdot D(c \circ \lambda_1)[t_1] =\)

\(D(f \circ c^{-1})[(c(\vec{p})] \cdot D(c \circ \lambda_1)[t_1]\)

which is the first term in the previous expression (Equation \ref{eq:addition-close}) (combining the
simplified left and right sides).

Similarly the second term
\(D(f \circ c^{-1})[c(\vec{p})] D(c \circ \lambda_2)[t_2])\) can be
shown by reversing the chain rule to be
\(D(f \circ \lambda_2)[t_2] \therefore\) the equation that shall remain
unnumbered can be reduced to:

\(D(f \circ \lambda_1)[t_1] + D(f \circ \lambda_2)[t_2]\)

which proves closure under addition of velocity vectors in the tangent space. We can verify the above terms for our concrete example:\\
\begin{sagesilent}
fl1 = f.substitute(x=K_1[0], y=K_1[1], z=K_1[2]).simplify_full()
fl2 = f.substitute(x=Tm[0], y=Tm[1], z=Tm[2]).simplify_full()
fl3 = f.substitute(x=K3[0], y=K3[1], z=K3[2]).simplify_full()
vv1 = fl1.diff(t).substitute(t=pt).n(digits=7)
vv2 = fl2.diff(s).substitute(s=0).n(digits=7)
vv3 = fl3.diff(t).substitute(t=0).n(digits=7)
\end{sagesilent}
$f \circ \lambda_{1,2,3} = \sage{fl1.substitute(t=pt).n(digits=7)}, \sage{fl2.substitute(s=0).n(digits=7)}, \sage{fl3.substitute(t=0).n(digits=7)}$\\
$D(f \circ \lambda_1) + D(f \circ \lambda_2) = \sage{vv1} + \sage{vv2} = \sage{vv1+vv2} = D(f \circ \lambda_3) = \sage{vv3}$
    
\hypertarget{scalar-multiplication-closure}{%
\subsubsection{Scalar Multiplication
Closure}\label{scalar-multiplication-closure}}

Proving closure under scalar multiplication is easier and does not
involve any charts. Again start with a construction for the new curve
\(\lambda_3\) tangent at \(\vec{p}\) ie
\(\lambda_1(t_1) = \lambda_3(t_3) = \vec{p}\) in terms of \(\lambda_1\):

\(\lambda_3(t) = \lambda_1(a t + t_1) = (\lambda_1 \circ u_a)[t_3]~~a \in R,~u_a(t) :R \rightarrow R = a t + t_1\)

At \(t = 0\) this evaluates to
\(\lambda_3(0) = \lambda_1(0 \cdot a + t_1) = \lambda_1(t_1) = \vec{p}\)
so \(t_3 = 0\).

\(D(f \circ \lambda_3)[t_3] = D(f \circ \lambda_3)[0] = D((f \circ \lambda_1) \circ u_a)[0]\)

Applying the chain rule:

\(D(f \circ \lambda_1)[u_a(0)] \cdot D(u_a)[0] = D(f \circ \lambda_1)[t_1] \cdot a = a D(f \circ \lambda_1)[t_1]\)

which proves closure under scalar multiplication.

    \hypertarget{chart-realizations-for-tangent-space}{%
\subsection{Chart Realizations for Tangent
Space}\label{chart-realizations-for-tangent-space}}

We start by introducing a chart \(c\) into a velocity vector applied to
an arbitrary function \(f\) for a curve
\(\lambda(t_1) = \vec{p} \in U \subseteq M\):

\(D((f \circ c^{-1}) \circ (c \circ \lambda))[t_1]\)

Applying the chain rule once again:

\(D_{u,v}(f \circ c^{-1})[c \circ \lambda(t_1)] \cdot D_t(c \circ \lambda)[t_1] = D_t(c \circ \lambda)[t_1] \cdot D_{u,v}(f \circ c^{-1})[c(\vec{p})]\)

The representation of derivation based tangent space vectors w.r.t
charts as used above in the proofs is quite complicated, so a simplified
representation, otherwise known as syntactic sugar in the computer
science community, is conventionally used, particularly in Physics,
which is shown here along with an example.

We start by investigating what the left and right hand sides of the
above look like for a concrete example using the north chart of the
stereographic projection and the spiral curve on the sphere. To do this
we take the Jacobian of both sides and then do matrix multiplication (or
dot product as they are both vectors) on the jacobians:\\
\begin{sagesilent}
t1 = vector(SR, (cn_0.substitute(x=K_1[0], z=K_1[2]), cn_1.substitute(y=K_1[1], z=K_1[2])))
lhs = jacobian(t1, (t)).transpose()

_pp = vector(SR, (cn_0.substitute(x=Pn[0], z=Pn[2]), cn_1.substitute(y=Pn[1], z=Pn[2])))
_f = f.substitute(x=cn_ix, y=cn_iy, z=cn_iz).simplify_full()
rhs = jacobian(_f, [u,v])
res = (lhs.substitute(t=pt)*rhs.transpose().substitute({u:_pp[0], v:_pp[1]})).n(digits=6)
\end{sagesilent}
\begin{multline*}
    \scriptstyle LHS = 
\left(\begin{array}{rr}
    \scriptstyle 
    \frac{7 \, \pi \sqrt{-t^{2} + 1} \sin\left(7 \, \pi t\right)}{t - 1} + \frac{t \cos\left(7 \, \pi t\right)}{\sqrt{-t^{2} + 1} {\left(t - 1\right)}} + \frac{\sqrt{-t^{2} + 1} \cos\left(7 \, \pi t\right)}{{\left(t - 1\right)}^{2}} & -\frac{7 \, \pi \sqrt{-t^{2} + 1} \cos\left(7 \, \pi t\right)}{t - 1} +\\ 
    \frac{t \sin\left(7 \, \pi t\right)}{\sqrt{-t^{2} + 1} {\left(t - 1\right)}} + \frac{\sqrt{-t^{2} + 1} \sin\left(7 \, \pi t\right)}{{\left(t - 1\right)}^{2}}
    \end{array}\right)
    \Big[t = \sage{pt} \Big]
\end{multline*}

$\scriptstyle RHS = \sage{rhs.transpose()}\Big[(u,v) = \sage{_pp.n(digits=6)} \Big]$
    
The right hand side of the expression is simplified as follows: 
\begin{enumerate}[itemsep=1pt, topsep=1pt, partopsep=0pt]
    \item As the derivative is being taken Leibnitz notation is used (odd - one would have thought physicists would prefer Newton), the D is replaced with a
    del ie \(\partial\),
    \item The function is used in the numerator and the chart is used in the denominator in the Leibnitz notation for differentiation 
    i.e.~\(\frac{\partial f}{\partial c}\),
    \item The coordinate index of the component being differentiated in the vector is used as a superscript in the denominator (the use of vectors and
    matrices was implied in the Jacobian differentiation operator used in most of this document) \(\frac{\partial f}{\partial c^i}\),
    \item Finally the point on the manifold is given as a subscript i.e \((\frac{\partial f}{\partial c^i})_p\).
\end{enumerate}
       
The left hand side is also simplified: 
\begin{enumerate}[itemsep=1pt, topsep=1pt, partopsep=0pt]
    \item As the curve is being differentiated with respect to a parameter which can be interpreted as time, Newton finally arrives on the scene, 
    with the Newtonian notation with a dot on top of the curve being used i.e.~\(\dot \lambda\),
    \item As with the RHS the component index is used as a superscript (which also ties in with using Einstein summation for the entire expression)
    i.e.~\(\dot \lambda^i\),
    \item The curve is used as a subscript ie \(\dot \lambda^i_c\) 4. Finally the point in the manifold is added i.e.~\(\dot \lambda^i_c(t_1)\)
\end{enumerate}    

The entire shorthand expression for a component is expressed as an operator which operates on
\(C^\infty\) functions: \(\dot \lambda^i_c(t_1) (\frac{\partial }{\partial c^i})_pf\)

The \((\frac{\partial }{\partial c^i})_p\) part (or more accurately its full expansion) forms a
basis for the vector space of derivations defining the tangent space.
\Textcite[see][pp.~82-86]{isham} provides a proof of this as well as the isomorphism between the
derivation and the previously described geometric tangent space. This basis is known as a chart
induced basis.

The example shows the basis dependent on \(u,v\) and the components on
\(t\):\\
\begin{multline*}
    \left(\begin{array}{r}
        \scriptstyle
        4 \, {\left(\frac{7 \, \pi \sqrt{-t^{2} + 1} \sin\left(7 \, \pi t\right)}{t - 1} + \frac{t \cos\left(7 \, \pi t\right)}{\sqrt{-t^{2} + 1} {\left(t - 1\right)}} + \frac{\sqrt{-t^{2} + 1} \cos\left(7 \, \pi t\right)}{{\left(t - 1\right)}^{2}}\right)} {\left(\frac{u + 1}{u^{2} + v^{2} + 1} - \frac{{\left(u^{2} + v^{2} + 2 \, u + 2 \, v - 1\right)} u}{{\left(u^{2} + v^{2} + 1\right)}^{2}}\right)} -\\
        \scriptstyle
        4 \, {\left(\frac{7 \, \pi \sqrt{-t^{2} + 1} \cos\left(7 \, \pi t\right)}{t - 1} - \frac{t \sin\left(7 \, \pi t\right)}{\sqrt{-t^{2} + 1} {\left(t - 1\right)}} - \frac{\sqrt{-t^{2} + 1} \sin\left(7 \, \pi t\right)}{{\left(t - 1\right)}^{2}}\right)} {\left(\frac{v + 1}{u^{2} + v^{2} + 1} - \frac{{\left(u^{2} + v^{2} + 2 \, u + 2 \, v - 1\right)} v}{{\left(u^{2} + v^{2} + 1\right)}^{2}}\right)}
        \end{array}\right)
        = \sage{res[0]}
\end{multline*}    
    
\hypertarget{change-of-basis-for-tangent-space-components}{%
\subsection{Change of basis for Tangent Space
Components}\label{change-of-basis-for-tangent-space-components}}

Given two charts \((U, c_1)\) and \((V, c_2)\) where
\(U \cap V \neq \emptyset\), where the tangent vector to a point
\(\vec{p} \in U,V\) is known relative to say \(U\), then the equivalent
basis vector for \(V\) can be derived:

\begin{enumerate}[itemsep=1pt, topsep=1pt, partopsep=0pt]
\item
  Recall the expression for the basic component from the previous
  section and once again apply the idiom of using the inverse of a map
  after the map, in this case the map being the chart from \(V\).
  \((\frac{\partial f}{\partial c_1^{i:1\cdots m}})_p = D(f \circ c_1^{-1})[c_1(\vec{p})] = D((f \circ c_2^{-1}) \circ (c_2 \circ c_1^{-1})[c_1(\vec{p})]\)
\item
  Utilise the chain rule again:
  \(D_{u,v}(f \circ c_2^{-1})[c_2 \circ c_1^{-1}(c_1(\vec{p})] \cdot D_{u,v}(c_2 \circ c_1^{-1})[c_1(\vec{p})] = D_{u,v}(c_2 \circ c_1^{-1})[c_1(\vec{p})] \cdot D_{u,v}(f \circ c_2^{-1})[c_2(\vec{p})]\).
\item
  The full expression for the tangent vector in \(U\) is given by:
  \(\dot \lambda^i_{c_1}(t_1) (\frac{\partial f}{\partial c_1^{i:1\cdots m}})_p = D(c_1 \circ \lambda)[t_1] \cdot D(f \circ c^{-1}_1)[c_1(\vec{p})]\)
  Substituting the expression from 2:
  \(D(c_1 \circ \lambda)[t_1] \cdot D(f \circ c^{-1})[c(\vec{p})] = D_t(c_1 \circ \lambda)[t_1] \cdot D_{u,v}(c_2 \circ c_1^{-1})[c_1(\vec{p})] \cdot D_{u,v}(f \circ c_2^{-1})[c_2(\vec{p})]\)
\item
  The full expression for the tangent vector in \(V\) is given by:
  \(\dot \lambda^i_{c_2}(t_2) (\frac{\partial f}{\partial c_2^{i:1\cdots n}})_p = D(c_{2} \circ \lambda)[t_2] \cdot D(f \circ c^{-1}_{2})[c_2(\vec{p})]\)
\item
  As the curve on the manifold remains the same, the tangent vector is
  also the same, its just expressed differently in the two bases,
  therefore these expressions can be equated:
  \(D_t(c_{2} \circ \lambda)[t_2] \cdot D_{u,v}(f \circ c^{-1}_{2})[c_2(\vec{p})] = D_t(c_1 \circ \lambda)[t_1] \cdot D_{u,v}(c_2 \circ c_1^{-1})[c_1(\vec{p})] \cdot D_{u,v}(f \circ c_2^{-1})[c_2(\vec{p})]\)
  The components in V are thus
  \(D_t(c_1 \circ \lambda)[t_1] \cdot D_{u,v}(c_2 \circ c_1^{-1})[c_1(\vec{p})]\)
\end{enumerate}

The \(D_{u,v}(c_2 \circ c_1^{-1})[c_1(\vec{p})]\) (Jacobian of the
overlap function between \(U\) and \(V\)) is \(R^m \rightarrow R^n\) so
the Jacobian will be a \(n{\times}m\) matrix. This matrix is multiplied
by a \(m{\times}1\) vector of the map from the curve to chart
coordinates via the manifold.

Applying this to the example using a change of basis from the north to
the south hemisphere of the stereographic projection (the point at which
the tangent are found is the northern hemisphere test point, and the
curve is the same as the one used in the previous tangent space
examples):\\
\begin{sagesilent}
pp1 = vector(SR, (cn_0.substitute(x=Pn[0], z=Pn[2]), cn_1.substitute(y=Pn[1], z=Pn[2])))
pp2 = vector(SR, (cs_0.substitute(x=Pn[0], z=Pn[2]), cs_1.substitute(y=Pn[1], z=Pn[2])))
fc2 = f.substitute(x=cs_ix, y=cs_iy, z=cs_iz).simplify_full()
c2c1 = vector(SR, (cs_0.substitute({x: cn_ix, z: cn_iz}).full_simplify(), cs_1.substitute({y: cn_iy, z: cn_iz}).full_simplify()))
c1l = vector(SR, (cn_0.substitute(x=K_1[0], z=K_1[2]), cn_1.substitute(y=K_1[1], z=K_1[2])))

Dc2c1 = jacobian(c2c1, (u, v))
Dc1l = jacobian(c1l, (t))
pretty_print(Dc2c1, LatexExpr("\\cdot"), Dc1l, LatexExpr("~=~"), Dc2c1*Dc1l)
pretty_print(Dc2c1.substitute(u=pp1[0], v=pp1[1]).n(digits=7), LatexExpr("\\cdot"), Dc1l.substitute(t=float(pt)).n(digits=7), LatexExpr("~=~"), (Dc2c1.substitute(u=pp1[0], v=pp1[1])*Dc1l.substitute(t=float(pt))).n(digits=7))    
\end{sagesilent}
% \begin{multline*}
% \scriptstyle 
% \sage{Dc2c1} \cdot \sage{Dc1l} =\\
% \sage{(Dc2c1.substitute(u=pp1[0], v=pp1[1])*Dc1l.substitute(t=float(pt))).n(digits=7)}
% \end{multline*}

\begin{multline*}
    \hspace{-1.6cm}
    \medmuskip=2mu
    \scriptstyle 
    \sage{Dc2c1} \cdot \sage{Dc1l}[t=\sage{float(pt)}] = \\
    \scriptstyle \sage{(Dc2c1*Dc1l.substitute(t=pt))}
\end{multline*}

\begin{multline*}
    \hspace{-1.6cm}
    \medmuskip=2mu
    \scriptstyle \sage{(Dc2c1*Dc1l.substitute(t=pt))}\Big[\begin{array}{c} u = \sage{pp1[0].n(digits=5)} \\ v = \sage{pp1[1].n(digits=5)}\end{array}\Big] =\\
    \scriptstyle \sage{(Dc2c1.substitute(u=pp1[0], v=pp1[1])*Dc1l.substitute(t=float(pt))).n(digits=7)}
\end{multline*}
    
    We can test the change of basis by calculating the component representation in the south chart directly:\\
\begin{sagesilent}    
lhss = vector(SR, (cs_0, cs_1))
lhs = jacobian(vector(SR, (cs_0.substitute(x=K_1[0], z=K_1[2]), cs_1.substitute(y=K_1[1], z=K_1[2]))), (t)).transpose()
\end{sagesilent}  
\begin{multline*}
    \hspace{-1.6cm}
    \medmuskip=2mu  
    D_{x,y,z}\sage{lhss}\Big[\begin{array}{c} x = \sage{K_1[0]} \\ y = \sage{K_1[1]} \\ z = \sage{K_1[2]}\end{array}\Big] =\\ 
    \hspace{-1.6cm}
    \scriptstyle \sage{lhs}\big[t=\sage{float(pt)}\big] =\\ 
    \scriptstyle \sage{lhs.substitute(t=pt).n(digits=6)}
\end{multline*}    
    
Most (physics) textbooks use a shorthand notation for \(D_{u,v}(c_2 \circ c_1^{-1})[c_1(\vec{p})]
\cdot D_t(c_1 \circ \lambda)[t_1]\), namely \(\frac{\partial c_2^j}{\partial c_1^i} \cdot
X^i_{c_1}\) where the \(\frac{\partial c_2^j}{\partial c_1^i}\) refers to the Jacobian matrix with
the superscript indices corresponding to rows and columns from the matrix, while the \(X^i_{c_1}\)
term is a generic reference to the components in the convert-from chart minus any curve
information. 

\printbibliography    
\end{document}
